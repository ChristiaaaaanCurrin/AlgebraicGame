\documentclass{article}

\usepackage{tikz}
\usepackage{amsmath}
\usepackage{geometry}
\usepackage{amssymb}
\usepackage{amsthm}
\usepackage{mathtools}

\title{Generalized Formalization of Games}
\author{Christiaan van de Sande \and Tanner Reese}

\theoremstyle{definition}
\newtheorem{definition}{Definition}[subsection]

\theoremstyle{plain}
\newtheorem{theorem}{Theorem}

\def\Rule{\mathcal{R}}
\DeclareMathOperator{\imp}{imp}

\newcommand{\depp}{\mathbin{\: \, \mathclap{\tikz{
  \draw (-.20em, -.20em) -- (.20em, .20em);
  \draw (-.20em,  .20em) -- (.20em, .20em);
  \draw ( .20em, -.20em) -- (.20em, .20em);
}}\: \,}}
 
\newcommand{\indp}{\mathbin{\: \,\mathclap{\tikz{
  \draw (0, 0) -- (0, -.25em);
  \draw (0, 0) -- (-.25em, .25em);
  \draw (0, 0) -- (.25em, .25em);
}}\: \,}}

\newcommand{\gate}{\mathopen{\; \mathclap{\tikz{
  \draw (0, -0.35em) -- (0, 0.30em);
  \draw (0,  0.30em) -- (0.30em, 0.50em);
}}\; \!}}

\begin{document}

\maketitle

\section{Introduction}

\section{Games}

\section{Moves}

\section{Rules}

\subsection{Motivation and Definition} % SUBSECTION

\begin{definition}
  Let $ A $ be a set and
  let $ (B, \circ) $ be a group with action on $ A $.
  Let $ r : A \rightarrow \mathcal{P} (B) $.
  Then $ r \in \Rule (A, B, \circ) $. 
\end{definition}

\subsection{Elementary Rules} % SUBSECTION

\begin{definition}
  Let $ (G, \cdot) $ be a Group with action on $ A $.
  Define $ r $ to be the rule \textbf{induced} by $ G $
  (written $ r = R (G, \cdot) $).
  Let $ x \in A $.
  Then $ r (x) = G $.
\end{definition}

\begin{definition}
  Let $ A $ and $ B $ be sets where $ A \subseteq B $.
  Let $ x \in B $.
  Then $ P_A \in \Rule (B, \{ e \}, \cdot) $
  where $ P_A = \begin{cases} \varnothing & x \notin A \\ \{ e \} & x \in A \end{cases} $
    and $ e $ denotes the identity element of the trivial group.
  For 
\end{definition}

\subsection{Operations on Rules} % SUBSECTION
\begin{align*}
              r & : A \rightarrow B \\
              s & : C \rightarrow D \\
     r \times s & : A \times C \rightarrow B \times D \\
    (r \times s) (x, y) & = r (x) \times s (y) \\
\end{align*}

\begin{align*}
              r & : A \rightarrow B \\
              s & : A \rightarrow C \\
     r \wedge s & : A \rightarrow B \cap C \\
    (r \wedge s) (x) & = r (x) \cap s (x) \\ 
\end{align*}

\begin{align*}
            r & : A \rightarrow B \\
            s & : A \rightarrow C \\
     r \vee s & : A \rightarrow B \cup C \\
    (r \vee   s) (x) & = r (x) \cup s (x) \\
\end{align*}

\begin{align*}
             r & : A \rightarrow B \\
             s & : A \rightarrow C \\
     r \cdot s & : A \rightarrow B \times C \\
    (r \cdot  s) (x) & = \{ z \circ y : y \in s (x), z \in r (y \cdot x) \} \\
\end{align*}

\begin{align*}
              r  & : A \rightarrow B \\
     \widehat{r} & : \mathcal{P} (A) \rightarrow A \times B \\
     \widehat{r} (X) & = \bigsqcup \{ r (x) : x \in X \} \\
\end{align*}

\begin{align*}
              n & \in \mathbb{N} \\
              r & : A \rightarrow B \\
            r^n & : A \rightarrow B \\
                 r^n & = \begin{cases}
                           P               & n = 0  \\
                           r \cdot r^{n-1} & n > 0
                         \end{cases} \\
\end{align*}

\begin{align*}
                   r & : A \rightarrow B \\
              r^{-1} & : A \rightarrow B \\
              r^{-1} & = \{ y^{-1} : y \in r (x) \} \\
\end{align*}

\begin{align*}
              r & : A \rightarrow B \\
              s & : C \rightarrow D \\
     r \times s & : A \times C \rightarrow B \sqcup D \\
             (r + s) & = (r \times P) \vee (P \times s)
\end{align*}

\subsection{Construction of Fundamental Rules}

\section{Evaluators}

\section{Conclusion}

\end{document}
