\documentclass{article}

\usepackage{tikz}
\usepackage{amsmath}
\usepackage{geometry}
\usepackage{amssymb}
\usepackage{amsthm} 

\title{Generalized Formalization of Games}
\author{Christiaan van de Sande \and Tanner Reese}

\theoremstyle{definition}
\newtheorem{definition}{Definition}[subsection]

\theoremstyle{plain}
\newtheorem{theorem}{Theorem}

\def\rule{\mathcal{R}}

\begin{document}

\maketitle

\section{Introduction}
 
\section{Games}

\section{Rules}

\subsection{Motivation and Definition} % SUBSECTION

The purpose of a rule is to define legal transitions between states in a game.
It may be tempting to define a rule as a function from one state to another state, but this approach is very limited,
as it does not help to describe the moves themselves, only the consequences of those moves.
The definition of a rule must include
the definition of a function $ \lambda $ that generates the legal moves from a given state
and the definition of a function $ \phi $ that executes a given move on a given state.
The moves must be independent of the state, so that the same move can be executed on different states.
Moves must also be reversable,
meaning that, given any states $ s $ and $ s' $ and move $ m $,
where $ m $ is the move from $ s $ to $ s' $,
there must be a way to find $ s $ from only $ s' $ and $ m $.
This reverse operation is the responsibility of a third function $ \rho $.
The requirements of these three functions gives the following definition for a rule:

 \begin{definition}
  $ r $ is a \textbf{rule} on sets $ A $ and $ B $ (written  $ r \in \rule (A, B) $) iff
  $ \lambda_r : A          \rightarrow \mathcal{P} (B) $,
  $    \phi_r : A \times B \rightarrow A $,
  $    \rho_r : A \times B \rightarrow A $,
  such that $ \phi_r (\rho_r (a, b), b) = a $
  and $ \rho_r (\phi_r (a, b), b) = a $,
  for all $ a \in A $, $ b \in B $.
\end{definition}

For the rule-of-play, $ r $, of a game,
the set of all well-formed states, $ S $, forms the input set (above denoted as $ A $) for $ r $
and the set of all well-formed moves, $ M $, forms the output set (above denoted as $ B $) for $ r $,
so $ r \in \rule (S, M) $.
However, not all rules have the set of well-formed states as their input set.
In fact, most do not.
For most rules the input set is the state with some data added or removed based on related moves.
Take, for example, the rule for capturing in chess.
In most cases, captures are performed by moving one piece onto a square occupied by another piece
(the notable exception to this is en-passant capturing)
Thus, the legal captures are dependent, not only on the state (i.e. where all the pieces are located),
but also on the move being made, a piece only captures on the square that it moves to.

It is \emph{possible} to define a function that finds all legal captures
and a second function that finds all legal moves that are not captures,
but this approach sacrifices the flexibility afforded by having
one rule that is responsible for the \emph{movement} of pieces
and having another rule that is responsible for the conditions under which pieces are \emph{captured}.
In the latter case, the idea of capturing on the square to which you move is separated from the idea of moving in certain ways.
That way chess movement patterns can be applied to games without capturing or with different systems for capturing
and the same capturing rule applies to all of the pieces in chess (except for pawns).

\subsection{Properties of rules} % SUBSECTION

In order to discuss rules,
it is necessary to establish some vocabulary.

\begin{definition} 
  Let $ r \in \rule (A, B) $. $ r $ is \textbf{repeatable} iff $ A \supseteq B $.
\end{definition}

\begin{definition}
  Let $ r \in \rule (A, A) $. $ r $ is \textbf{constant} iff $ \lambda_r (a) = \{ a \} $ for all $ a \in A $.
\end{definition}

Constant rules are rules that do not present the player with any choice as to how the rule is applied.
Thus, the legal moves from a constant rule on any given input is the singleton set containing only that input.
Notice that all constant rules are also repeatable.

\begin{definition}
  Let $ r \in \rule (A, B) $. $ r $ is \textbf{passive} iff $ \phi_r (a, b) = a $ for all $ a \in A $, $ b \in B $.
\end{definition}

\begin{definition}
  Let $ r \in \rule (A, A) $. $ r $ is \textbf{simple} iff $ \phi_r (a_1, a_2) = a_2 $ for all $ a_1, a_2 \in A $.
\end{definition}

Notice that all simple rules are also repeatable.

\begin{definition}
  Let $ r \in \rule (A, B) $
  and $ s \in \rule (A, C) $.
  $r$ and $s$ are \textbf{independent} ($ r \perp s $) iff,
  $ \phi_r (\phi_s (a, c), b) = \phi_s (\phi_r (a, b), c) $, 
  for all $ a \in A $, $ b \in B $, $ c \in C $.
\end{definition}

\begin{definition}
  Let $ r \in \rule (A, B) $ and $ s \in \rule (B, C) $.
  Then $ r $ is \textbf{precedes} $ s $
  ($ r \prec s $)
  and $ s $ \textbf{succeeds} $ r $
  ($ s \succ r $).
\end{definition}

\subsection{Operations on Rules} % SUBSECTION

While the rule-of-play for a game can be created using a single rule
that handles all off the intricacies of the game,
The advantage offered by the rule structure
is the ability to combine simpler rules,
each of which control only a small part of the game.
Simple rules can be combined into more complex ones using the following operations.

\begin{definition}
  Let $ r \in \rule (A, B) $
  and $ s $ be the \textbf{complement} of $ r $
  (written $ s = \overline{r} $).
  Then $ s \in \rule (A, A) $, where:
  \begin{align}
    \lambda_s (a)    & = \begin{cases}
                           \varnothing & \lambda_r (a) \neq \varnothing \\
                           \{ a \}     & \lambda_r (a) = \varnothing
                         \end{cases} \\
       \phi_s (a_1, a_2) & = a_2 \\
       \rho_s (a_1, a_2) & = a_2.
  \end{align}
\end{definition}

\begin{definition}
  Let $ r \in \rule (A, B) $
  and $ s $ be the \textbf{reduction} of $ r $
  (written $ s = \overline{\overline{r}} $).
  Then $ s \in \rule (A, A) $, where:
  \begin{align}
    \lambda_s (a)    & = \begin{cases}
                           \varnothing & \lambda_r (a) = \varnothing \\
                           \{ a \}     & \lambda_r (a) \neq \varnothing
                         \end{cases} \\
       \phi_s (a_1, a_2) & = a_2 \\
       \rho_s (a_1, b_2) & = a_2.
  \end{align}
\end{definition}

Notice that the reduction of $ r $ is equivalent to the complement of the complement of $ r $
and that both the reduction of $ r $ and the complement of $ r $ are both simple and constant for any $ r \in \rule (A, B) $.

\begin{definition}
  Let $ r \in \rule (A, B) $
  and $ s \in \rule (A, C) $
  and $ \phi_s (a, b) = \phi_r (a, b) $
  for all $ a \in A, b \in B \cap C $
  and $ t $ be the \textbf{union} of $ s $ and $ r $
  (written $ t = s \cup r $).
  Then $ t \in \rule (A, B \cup C) $, where:
  \begin{align}
    \lambda_t (a)    & = \lambda_r (a) \cup \lambda_s (a) \\
       \phi_t (a, b) & = \begin {cases}
                           \phi_s (a, b)                 & b \in C \setminus B \\
                           \phi_r (a, b)                 & b \in B \setminus C \\
                           \phi_s (a, b) = \phi_r (a, b) & b \in B \cap C
                         \end{cases} \\
        \rho_t (a, b) & = \begin {cases}
                           \rho_s (a, b)                 & b \in C \setminus B \\
                           \rho_r (a, b)                 & b \in B \setminus C \\
                           \rho_s (a, b) = \rho_r (a, b) & b \in B \cap C
                         \end{cases}. 
  \end{align}
\end{definition}

\begin{definition}
  Let $ r \in \rule (A, B) $
  and $ s \in \rule (A, C) $
  and $ \phi_s (a, b) = \phi_r (a, b) $
  for all $ a \in A, b \in B \cap C $
  and $ t $ be the \textbf{intersection} of $ s $ and $ r $
  (written $ t = s \cap r $).
  Then $ t \in \rule (A, B \cap C) $, where:
  \begin{align}
    \lambda_t (a)    & = \lambda_r (a) \cap \lambda_s (a) \\
       \phi_t (a, b) & = \phi_s (a, b) = \phi_r (a, b) \\
       \rho_t (a, b) & = \rho_s (a, b) = \rho_r (a, b).
  \end{align}
\end{definition}

\begin{definition}
  Let $ r \in \rule (A, B) $
  and $ s \in \rule (A, C) $
  and $ t $ be the \textbf{independent product} of $ s $ and $ r $
  (written $ t = s \times r $).
  Then $ t \in \rule (A, B \times C) $, where:
  \begin{align}
    \lambda_t (a)         & = \lambda_r (a) \times \lambda_s (a) \\
       \phi_t (a, (b, c)) & =    \phi_r (\phi_s (a, c), b) \\
       \rho_t (a, (b, c)) & =    \rho_s (\rho_r (a, b), c).
  \end{align}
\end{definition}

\begin{definition}
  Let $ r \in \rule (A, B) $
  and $ s \in \rule (B, C) $
  and $ t $ be the \textbf{dependent product} of $ s $ and $ r $
  (written $ t = s \cdot r $).
  Then $ t \in \rule (A, B \times C) $, where:
  \begin{align}
    \lambda_t (a)         & = \{ (b, c) | b \in \lambda_r (a), c \in \lambda_s (b) \} \\
       \phi_t (a, (b, c)) & =    \phi_r (a, \phi_s (b, c)) \\
       \rho_t (a, (b, c)) & =    \rho_r (a, \rho_s (b, c)).
  \end{align}
\end{definition}

\begin{definition}
  Let $ r \in \rule (A, B) $
  and $ s \in \rule (C, D) $
  and $ t $ be the \textbf{full product} of $ s $ and $ r $
  (written $ t = s \bowtie r $).
  Then $ t \in \rule (A \times B, C \times D) $, where:
  \begin{align}
    \lambda_t ((a, c))         & = \{ (b, d) | b \in \lambda_r (a), d \in \lambda_s (c) \} \\
       \phi_t ((a, c), (b, d)) & = (\phi_r (a, b), \phi_s (c, d)) \\
       \rho_t ((a, c), (b, d)) & = (\rho_r (a, b), \rho_s (c, d)).
  \end{align}
\end{definition}

\begin{definition}
  Let $ r \in \rule (B, B) $
  and $ s \in \rule (A, B) $
  and $ t \in \rule (B, C) $
  and $ v $ be $ r $ \textbf{patterned} from $ s $ to $ t $
  (written $ v = r \rvert_{s}^{t} $).
  Then, $ v \in \rule (A, B) $, where:
  \begin{align}
      \kappa_v (b) & = \begin{cases}
                         \{ b \}                                               & \lambda_t (b) \neq \varnothing \\
                         \{ b \} \cup (\widehat{\kappa}_v \circ \lambda_r (b)) & \lambda_t (b) =    \varnothing
                       \end{cases} \\
  \lambda_v (a)    & = \widehat{\kappa}_v \circ \lambda_s (a) \\
     \phi_v (a, b) & = \phi_s (a, \phi_r (b, b)) \\
     \rho_v (a, b) & = \rho_s (a, \rho_r (b, b)) 
  \end{align}
\end{definition}

\subsection{Elementary Rules} % SUBSECTION

\begin{definition}
  Let $ r $ be the \textbf{void rule}
  (written $ r = \rule_z $).
  Then $ r \in \rule (A, B) $, where:
  \begin{align}
    \lambda_r (a)    & = \varnothing \\
       \phi_r (a, b) & = a \\
       \rho_r (a, b) & = a.
  \end{align}
\end{definition}

Notice that the void rule is passive and constant.

\begin{definition}
  Let $ r $ be the \textbf{pass rule}
  (written $ r = \rule_0 $).
  Then $ r \in \rule (A, A) $ where:
  \begin{align}
    \lambda_r (a)        & = {a} \\
       \phi_r (a_1, a_2) & =  a_2 \\
       \rho_r (a_1, a_2) & =  a_2.
  \end{align}
\end{definition}

Notice that the pass rule is simple and constant.

\begin{definition}
  Let $ n \in \mathbb{N} $ such that $ n \neq 0 $).
  Let $ r $ be the \textbf{extractor} of order $ n $
  (written $ r = \rule_n $).
  Then $ \rule_1 \in \rule (A \times B, A) $ where:
  \begin{align}
    \lambda_{\rule_1} ((a, b))        & = {a} \\
       \phi_{\rule_1} ((a_1, b), a_2) & = (a_2, b) \\
       \rho_{\rule_1} ((a_1, b), a_2) & = (a_2, b)
  \end{align}
\end{definition}

Notice that the extractor of any order is constant, but not simple.

\section{Evaluators}

\section{Conclusion}

\end{document}
