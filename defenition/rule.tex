\documentclass{article}

\usepackage{tikz}
\usepackage{amsmath}
\usepackage[margin=0.75in]{geometry}
\usepackage{amssymb}
\usepackage{amsthm} 

\title{Generalized Formalization of Game Rules}
\author{Christiaan van de Sande \and Tanner Reese}

\theoremstyle{definition}
\newtheorem{definition}{Definition}[section]

\theoremstyle{plain}
\newtheorem{theorem}{Theorem}

\def\rule{\mathcal{R}}

\begin{document}

\maketitle

\section{Introduction}

\section{Rules}

\begin{definition}
  A rule $ (r) $ on sets $ A $ and $ B $ is denoted by $ r \in \rule (A, B) $,
  with $ \lambda_r : A          \rightarrow \mathcal{P} (B) $,
       $     \mu_r : A \times B \rightarrow B $,
       $    \phi_r : A \times B \rightarrow B $, 
  such that $ \mu_r (\phi_r (a, b), b) = a $
  and $ \phi_r (\mu_r (a, b), b) = a $.
\end{definition}

The functions of a rule are responsible for replicating the mechanics of a game.
The input set, $A$, is the set of all possible game states; the output set, $B$, is the set of all well formed moves that could apply to any game state.
Then, $\lambda$ maps a given game state to a set of legal moves on that game state;
$\mu$ maps a game state and a move to a new game state,
and $\phi$ reverts the change made by $\mu$.

\section{Properties of rules}

\begin{definition}
  Let $ r \in \rule (A, B) $. $ r $ is \textbf{repeatable} iff $ A \subseteq B $.
\end{definition}

\begin{definition}
  Let $ r \in \rule (A, B) $
  and $ s \in \rule (A, C) $.
  $r$ and $s$ are \textbf{independent} ($ r \perp s $) iff,
  for all ($ a \in A $, $ b \in B $, $ c \in C $),
  $ \mu_r (\mu_s (a, c), b) = \mu_s (\mu_r (a, b), c) $. 
\end{definition}

\section{Operations on Rules}
\begin{definition}
  Let $ r \in \rule (A, B) $
  and $ s $ be the \textbf{reduction} of $ r $.
  Then $ s \in \rule (A, A) $, where:
  \begin{align}
            s        & = \overline{r} \\
    \lambda_s (a)    & = \begin{cases}
                           \varnothing & \lambda_r (a) = \varnothing \\
                           \{ a \}     & \lambda_r (a) \neq \varnothing
                         \end{cases} \\
        \mu_s (a, b) & = a \\
       \phi_s (a, b) & = a.
  \end{align}
\end{definition}

\begin{definition}
  Let $ r \in \rule (A, A) $
  and $ s $ be the \textbf{negation} of $ r $.
  Then $ s \in \rule (A, A) $, where:
  \begin{align}
            s        & = \neg r \\
    \lambda_s (a)    & = \begin{cases}
                           \varnothing & \lambda_r (a) \neq \varnothing \\
                           \{ a \}     & \lambda_r (a) = \varnothing
                         \end{cases} \\
        \mu_s (a, b) & =  \mu_r (a, b) \\
       \phi_s (a, b) & = \phi_r (a, b).
  \end{align}
\end{definition}

\begin{definition}
  Let $ r \in \rule (A, B) $
  and $ s \in \rule (A, B) $
  and $ t $ be the \textbf{union} of $ r $ and $ s $.
  Then $ t \in \rule (A, B) $, where:
  \begin{align}
            t        & = s \cup r \\
    \lambda_t (a)    & = \lambda_r (a) \cup \lambda_s (a) \\
        \mu_t (a, b) & = \mu_r  (a, b)    = \mu_s  (a, b) \\
       \phi_t (a, b) & = \phi_r (a, b)    = \phi_s (a, b).
  \end{align}
\end{definition}

\begin{definition}
  Let $ r \in \rule (A, B) $
  and $ s \in \rule (A, B) $
  and $ t $ be the \textbf{intersection} of $ r $ and $ s $.
  Then $ t \in \rule (A, B) $, where:
  \begin{align}
            t        & = s \cap r \\
    \lambda_t (a)    & = \lambda_r (a) \cap \lambda_s (a) \\
        \mu_t (a, b) & = \mu_r  (a, b)    = \mu_s  (a, b) \\
       \phi_t (a, b) & = \phi_r (a, b)    = \phi_s (a, b).
  \end{align}
\end{definition}

\begin{definition}
  Let $ r \in \rule (A, B) $
  and $ s \in \rule (A, C) $
  and $ t $ be the \textbf{independent product} of $ r $ and $ s $.
  Then $ t \in \rule (A, B \times C) $, where:
  \begin{align}
            t             & = s \times r \\
    \lambda_t (a)         & = \lambda_r (a) \times \lambda_s (a) \\
        \mu_t (a, (b, c)) & =     \mu_r ( \mu_s (a, c), b) \\
       \phi_t (a, (b, c)) & =    \phi_s (\phi_r (a, b), c).
  \end{align}
\end{definition}

\begin{definition}
  Let $ r \in \rule (A, B) $
  and $ s \in \rule (B, C) $
  and $ t $ be the \textbf{dependent product} of $ r $ and $ s $.
  Then $ t \in \rule (A, B \times C) $, where:
  \begin{align}
            t             & = s \cdot r \\
    \lambda_t (a)         & = \{ (b, c) | b \in \lambda_r (a), c \in \lambda_s (b) \} \\
        \mu_t (a, (b, c)) & =     \mu_r ( \mu_s (a, c), b) \\
       \phi_t (a, (b, c)) & =    \phi_s (\phi_r (a, b), c).
  \end{align}
\end{definition}

\begin{definition}
  Let $ r \in \rule (B, B) $
  and $ s \in \rule (A, B) $
  and $ t \in \rule (A, C) $
  and $ v $ be $ r $ \textbf{patterned} from $ s $ to $ t $.
  Then, $ v \in \rule (A, B) $, where:
  \begin{align}
          v        & = r \rvert_{s}^{t} \\
      \nu_v (b)    & = \begin{cases}
                         \{ b \}                                            & \lambda_t (b) \neq \varnothing \\
                         \{ b \} \cup (\widehat{\nu}_v \circ \lambda_r (x)) & \lambda_t (b) =    \varnothing
                       \end{cases} \\
  \lambda_v (a)    & = \widehat{\nu}_v \circ \lambda_s (a) \\
      \mu_v (a, b) & =  \mu_s (a,  \mu_r (b, b)) \\
     \phi_v (a, b) & = \phi_s (a, \phi_r (b, b)) 
  \end{align}
\end{definition}

\section{Implementation}

\section{Implications}

\subsection{Game Creation}

\subsection{Game Analysis}

\section{Conclusion}

\end{document}
